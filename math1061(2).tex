\documentclass{article}

\usepackage{amsmath}
\usepackage{amssymb}
\usepackage{graphicx}
\title{MATH1061 Course Notes}
\author{Paddy Maher}

\begin{document}
\maketitle
\newpage

\section{Logic}
\subsection{Logical Connectives}
\subsubsection{Basic logical connectives}
For a given logical statement come logical connectives. Basic logical connectives include:
\begin{itemize}
\item \textbf{not} = $\sim$
\item \textbf{and} = $\wedge$
\item \textbf{or} = $\lor$
\item \textbf{exclusive or} = $\oplus$
\end{itemize}

\subsubsection{Logical Equivalence}
Given two statement forms, you can show that they are logically equivalent by using a truth table or by using 
the laws of logical equivalence. \\
The logical equivalence between two statements is demonstrated by the symbol $\equiv$

[insert truth table]

\subsubsection{Conditional logical connectives}
Logical connectives and equivalences, for given truth statements '\textit{p}' and '\textit{q}'
\begin{itemize}
\item \textbf{if .. then} = $\rightarrow$
\item \textbf{if and only if} = $\leftrightarrow$
\item \textit{p} $\rightarrow$ \textit{q} $\equiv$ 
$\sim$ \textit{p} $\lor$ \textit{q} $\equiv$ 
$\sim$ \textit{q} $\rightarrow$ $\sim$ \textit{p}
\end{itemize}

\subsubsection{Order of Operations}
\begin{enumerate}
\item $\sim$
\item $\wedge$ and $\lor$, use parentheses to specify. If no parentheses given, work from left to right.
\item  $\rightarrow$ and $\leftrightarrow$, use parentheses to specify. If no parentheses given, work from left to right.
\end{enumerate}

\subsection{Necessary and sufficient conditions}
For given truth statements '\textit{p}' and '\textit{q}':
\begin{itemize}
\item \textit{p} is a necessary condition for \textit{q} means "if $\sim$\textit{p} then $\sim$\textit{q}" or
equivalently "if \textit{q} then \textit{p}" or "\textit{q} only if \textit{p}".
\item \textit{p} is a sufficient condition for \textit{q} means "if \textit{p} then \textit{q}" or equivalently
"\textit{q} if \textit{p}"
\end{itemize}

\subsection{Definitions}
\subsubsection{Tautology and contradictions}
\begin{itemize}
\item A \underline{tautology} is a statement form which always takes truth values "\textbf{true}" for all possible
truth values of its variables.
\item A \underline{contradiction} is a statement form which always takes truth values "\textbf{false}" for all possible
truth values of its variables.
\end{itemize}

\subsubsection{Contrapositive}
For given truth statements '\textit{p}' and '\textit{q}' \\
The \underline{contrapositive} of \textit{p} $\rightarrow$ \textit{q} is $\sim$\textit{q} $\rightarrow$ $\sim$\textit{p}
\begin{itemize}
\item These are logically equivalent: \\
\textit{p} $\rightarrow$ \textit{q} $\equiv$  $\sim$\textit{q} $\rightarrow$ $\sim$\textit{p}
\end{itemize}

\subsubsection{Biconditional}
For given truth statements '\textit{p}' and '\textit{q}' \\
The \underline{biconditional} of  \textit{p} and \textit{q}, denoted \textit{p} $\leftrightarrow$ \textit{q},
 is defined by the following truth table:

[insert truth table]

\subsection{Arguments}

\subsubsection{Premises}

Given a collection of statements '$p_1 , p_2, ... , p_n$' (called \textbf{premises}) and another statement 'q' (called the conclusion),
an '\textit{argument}' is the assertion that the conjunction of the premisees implies the conclusion.

\begin{align*}
& p_1 \\ 
& p_2 \\
& ... \\
& p_n \\
& \therefore q
\end{align*}

\subsubsection{Arguments; validity and invalidity}

\underline{Definition; valid argument}
An argument is \textbf{valid} if whenever all of the premises are true, the conclusion is also true. 

Thus, an argument is valid if ($p_1 \wedge p_2 \wedge ... \wedge p_n \rightarrow q$) is a tautology.

\underline{Definition; invalid argument}
An argument is \textbf{invalid} if it is possible to have a situation in which all of the premises are true but the conclusion is false.

We can check whether an argument is valid or invalid using a truth table.

\subsubsection{Rules of Inference}

Modus Ponens 
\begin{align*}
& p \rightarrow q \\ 
& p \\
& \therefore q
\end{align*}

Modus Tollens
\begin{align*}
& p \rightarrow q \\ 
& \sim q \\
& \therefore \sim p
\end{align*} \\

Generalisation
\begin{align*}
& p \\ 
& \therefore p \lor q \\ \\
& q \\ 
& \therefore p \lor q 
\end{align*} \\

Specialisation
\begin{align*}
& p \wedge q \\ 
& \therefore p \\ \\
& p \wedge q \\ 
& \therefore q 
\end{align*} \\

Conjunction
\begin{align*}
& p \\ 
& q \\
& \therefore p \wedge q
\end{align*} \\

Elimination
\begin{align*}
& p \lor q \\ 
& \sim q \\
& \therefore p \\ \\
& p \lor q \\ 
& \sim p \\
& \therefore q
\end{align*} \\

Transitivity
\begin{align*}
& p \rightarrow q \\ 
& q \rightarrow r \\
& \therefore p \rightarrow r
\end{align*} \\

Proof by Division into cases
\begin{align*}
& p \lor q \\ 
& p \rightarrow q \\
& q \rightarrow r \\
& \therefore r
\end{align*} \\

Contradiction Rule
\begin{align*}
& \sim p \rightarrow (contradiction) \\
& \therefore p
\end{align*} \\

\subsubsection{Alternative method for determining validity}

If an argument is \textit{'invalid'} then there is a situation where all the premises are true but the conclusion is false.

Attempt to see whether this is possible. To do this, look for truth values which make all premises true yet the conclusion is false.

If such truth values can be found, then the argument is \textit{invalid} \\

Summary of this method:
\begin{itemize}
\item Try to make all the premises true and the conclusion false
\item If this can be done, then the argument is \underline{invalid}
\item On the other hand, if this is \textbf{impossible} to do, then the argument is \underline{valid}
\end{itemize}

\underline{Checks for Validity}
\begin{itemize}
\item Use a truth table
\item Use rules of inference
\item Attempt to find turht values that make all premises true but the conclusion false.
\end{itemize}

\subsubsection{Predicates and domains}

A \underline{predicate} is a sentence that contains finitely many variables, and which becomes a statement if the variables are given
specific values. \\

The \underline{domain} of each variable in a predicate is the set of all possible values that may be assigned to it. \\

The \underline{truth set} of a predicate P(x) is the set of all values in the domain that, when assigned to x, make P(x) a true statement.

\underline{Common Domains}
\begin{itemize}
\item Integers: $\mathbb{Z}$ = [ ..., -3, -2, -1, 0, 1, 2, 3, .. ]
\item Positive integers: $\mathbb{Z}^{+}$ = [ 1, 2, 3, ... ]
\item Non-negative integers: $\mathbb{Z}^{non-neg}$ = [ 0, 1, 2, 3, ... ]
\item Natural numbers: $\mathbb{N}$ = [ 1, 2, 3, ... ]
\item Rational numbers: $\mathbb{Q}$ = [ $\frac{a}{b}$ $|$ a, b $\in \mathbb{Z} \wedge$ b $\neq$ 0 ]
\item Real numbers: $\mathbb{R}$ = entire number line
\end{itemize}

\subsection{Quantifiers}

\subsubsection{Universal and Existental quantifiers}

\underline{The Universal quantifier}

The symbol '$\forall$' denotes \textit{"for all"} and is called the \textbf{universal quantifier}

Let Q(x) be a predicate and \textbf{D} be the domain of x. The \textbf{universal statement}

$\forall$ x $\in$ \textbf{D}, Q(x)

is true if and only if Q(x) is true for every x in \textbf{D}. It is false if and only if Q(x) is false for at least one x in \textbf{D} \\

\underline{The Existental Quantifier}

The symbol '$\exists$' denotes "there exists" and is called the \textbf{existential quantifier}

Let Q(x) be a predicate and \textbf{D} be the domain of x. The \textbf{existential statement}

$\exists$ x $\in$ \textbf{D} such that Q(x)

is true if and only if Q(x) is true for at least one x in \textbf{D}. It is false if and only if Q(x) is false for every x in \textbf{D}

\subsubsection{Negation of Quantified Statements}

\underline{Universal Statement:}

$\forall$ x $\in$ \textbf{D}, Q(x)

The negation of this statement is logically equivalent to:

$\exists$ x $\in$ \textbf{D} such that $\sim$ Q(x) \\

\underline{Existential Statement:}

$\exists$ x $\in$ \textbf{D} such that Q(x) 

The negation of this statement is logically equivalent to:

$\forall$ x $\in$ \textbf{D}, $\sim$ Q(x) \\

\underline{Universal Conditional Statement}

$\forall$ x $\in$ \textbf{D} if P(x) then Q(x)

The negation of this statment is logically equivalent to:

$\exists$ x $\in$ \textbf{D} such that $\sim$ if P(x) then Q(x) 

which is;

$\exists$ x $\in$ \textbf{D} such that P(x) $\wedge \sim$Q(x)

\subsection{Multiple quantifiers}

\subsubsection{Intro to multiple quantifers}

The predicate x $\leq$ y for real numbers \textit{x} and \textit{y} involves more than one variable.

Notation such as P(x,y) is used to denote such predicates.

Such predicates often appear in statments that involve more than one quantifier \\

In order to establish the truth of a statment of the form:

$\forall$ x $\in$ \textbf{D}, $\exists$ y $\in$ \textbf{D} such that P(x,y)

One must allow another to pick whatever element x $\in$ \textbf{D} they wish, and then must preceed with finding an element
y $\in$ \textbf{E} which makes P(x,y) true. \\

In order to establish the truth of a statment of the form:

$\exists$ x $\in$ \textbf{D} such that $\forall$ y $\in$ \textbf{D}, P(x,y)

One must find one particular x $\in$ \textbf{D} which makes P(x,y) true no matter which y $\in$ \textbf{D} might be chosen
for you.

\subsubsection{Negation of statements with multiple quantifers}

The statement: 

$\forall$ x $\in$ \textbf{D}, $\exists$ y $\in$ \textbf{E} such that P(x,y)

Negates to: 

$\exists$ x $\in$ \textbf{D} such that $\sim$ ($\exists$ y $\in$ \textbf{E} such that P(x,y))

Which is:

$\exists$ x $\in$ \textbf{D} such that $\forall$ y $\in$ \textbf{E} $\sim$ P(x,y) \\


The statement: 

$\exists$ x $\in$ \textbf{D} such that $\forall$ y $\in$ \textbf{E}, P(x,y)

Negates to:

$\forall$ x $\in$ \textbf{D}, $\sim$ ($\forall$ y $\in$ \textbf{E}, P(x,y))

Which is:

$\forall$ x $\in$ \textbf{D}, $\exists$ y $\in$ \textbf{E} such that $\sim$ P(x,y))

\section{Proofs and Number Theory}

\subsection{Proofs}

\subsubsection{Even and Odd}

An integer 'n' is \underline{even} if and only if 'n' is twice some integer.
\begin{itemize}
\item That is: n is even $\leftrightarrow \exists k \in \mathbb{Z}$ such that n = 2k
\end{itemize}

An integer 'n' is \underline{odd} if and only if 'n' is twice some integer.
\begin{itemize}
\item That is: n is odd $\leftrightarrow \exists k \in \mathbb{Z}$ such that n = 2k + 1
\end{itemize}

\subsubsection{Prime and Composite}

An integer 'n' is \underline{prime} if and only if n $>$ 1 and for all positive integers 'r' and 's', if n = rs, then r = 1 or s = 1.

An integer 'n' is \underline{composite} if and only if n $>$ 1 and n = rs for some positive integers 'r' and 's' with r $\neq$ 1 and s $\neq$ 1

\underline{In symbols:}

\begin{itemize}
\item n is prime $\leftrightarrow$ n $>$ 1 $\wedge \forall$ r,s $\in \mathbb{Z}^{+}$, n = rs $\rightarrow$ ( r=1 $\wedge$ s=1 )
\item n is composite $\leftrightarrow$ n $>$ 1 $\wedge \exists$ r, s $\in \mathbb{Z}^{+}$ such that n = rs $\wedge$ r $\neq$ 1 $\wedge$ s $\neq$ 1
\item Note: 1 is neither prime nor composite.
\end{itemize}

\subsubsection{Direct proofs}

\underline{Proving Existental Statements}

To show $\exists$ x $\in$ \textbf{D} such that P(x) is true, it is enough to find one example of an element x in \textbf{D} for which P(x) is true. \\

\underline{Direct Proof of Universal Statements}

One way to show that $\forall$ x $\in$ \textbf{D}, P(x) is true is by a \underline{direct proof}.

\begin{enumerate}
\item Suppose x $\in$ \textbf{D}
\item Show that P(x) is true
\end{enumerate}

Method of direct proof to show that $\forall$ x $\in$ \textbf{D} if P(x) then Q(x) is true:
\begin{enumerate}
\item Suppose x $\in$ \textbf{D} and P(x) is true
\item Show that the conclusion Q(x) is true using definitions, previously established results, and the rules for logical inference.
\end{enumerate}

\underline{How to write a proof}

\begin{itemize}
\item Write the theorem to be proved
\item Clearly mark the beginning of the proof with the word \textit{"Proof"}
\item Use precise definitions for any mathematical terms
\item Write the proof in complete sentences
\item Give a reason for each assertion in your proof
\item Make your proof self-contained
\item Display equations and inequalities clearly
\item Conclude by stating what it is you have proved
\end{itemize}

\subsubsection{Other forms of proof}

\underline{Disproof by Counterexample}
To show that $\forall$ x $\in$ \textbf{D} if P(x) then Q(x) is false, find a value of x $\in$ \textbf{D} for which P(x) is true and Q(x) is false. \\ \\ 

\underline{Method of Proof by Contradiction}

\begin{enumerate}
\item Assume that the statement is false
\item Show that this leads logically to a contradiction
\begin{itemize}
\item So it is impossible that the statement is false
\end{itemize}
\item Conclude that the statement is true
\end{enumerate}

\underline{Method of Proof by Contradiction to show that $\forall$ x $\in$ \textbf{D}, if P(x) then Q(x) is true:}

\begin{enumerate}
\item Assume that the statement to be proved is false. Thus, the \textit{negation} of the statement is true: $\exists$ x $\in$ \textbf{D} such that P(x) and $\sim$ Q(x)
\item Show that this assumption leads logically to a contradiction
\item Conclude that the statement to be proved is true.
\end{enumerate}

\underline{Method of Proof by Contraposition}

\begin{enumerate}
\item Express the statement to be proved in the form: $\forall$ x $\in$ \textbf{D}, if P(x) then Q(x)
\item Rewrite this statement in the \underline{contrapositive} form: $\forall$ x $\in$ \textbf{D}, if $\sim$ Q(x) the $\sim$ P(x)
\item Prove the contrapositive by direct proof:
\begin{itemize}
\item Suppose x $\in$ \textbf{D} and Q(x) is false
\item Show that P(x) is false
\end{itemize}
\end{enumerate}

\subsection{Types of numbers}

\subsubsection{Rational numbers}

A real number is \underline{rational} if and only if it can be expressed as a quotient of two integers with a non-zero denominator.

The set of all rational numbers is denoted by $\mathbb{Q}$

A real number that is not rational is \underline{irrational}

r is rational $\leftrightarrow$ $\exists$ a, b $\in \mathbb{Z}$ such that r = $\frac{a}{b}$ and b $\neq$ 0

\underline{Fact:}
The decimal expansion of a rational number either repeats or terminates

\underline{Fact:}
The decimal expansion of an irrational number does not repeat and does not terminate

\underline{Theorem:}
For all rational numbers r and s where r $<$ s, there exists another rational number q such that r $<$ q $<$ s

\subsubsection{Divisibility}

If n,d $\in \mathbb{Z}$, d $\neq$ 0, then n is \underline{divisible} by d if and only if there exists some k $\in \mathbb{Z}$ such that n = kd

Alternatively, we say
\begin{itemize}
\item n is a \underline{multiple} of d
\item d is a \underline{factor} of n
\item d is a \underline{divisor} of n
\item d \underline{divides} n
\end{itemize}

The notation d$|$n is used to represent the predicate "d divides n"

If n is not divisible by d, we write d$\nmid$n

\underline{Theorem:}
Every integer n$>$1 can be written as a product of primes.

\underline{Proof:}
Suppose the theorem is false. Then there exists an integer n$>$1 that is not a product of primes.

Choose the smallest number \textit{n}. Either n is prime or n is composite

\begin{itemize}
\item If n is prime; then n is trivially a product of primes
\item If n is composite: then n = rs for some positive integers r and s, where r $\neq$ 1 and s $\neq$ 1. This implies that 1 $<$ r $<$ n and 1 $<$ s $<$ n
\end{itemize}

Because we chose n to be the smallest integer greater than 1 that is not a product of primes, both r and s (which are smaller than n) \underline{must} be products of primes.

Therefore, n = rs is a product of primes also.

So, regardless of whether n is prime or composite, we find that n is a product of primes.

This contradicts our choice of n

Therefore, every integer n $>$ 1 can be written as a product of primes.

\subsubsection{Prime Factorisation}

Given any integer n $>$ 1, there exists
\begin{itemize}
\item A positive integer \textit{k}
\item Distinct primes $p_{1}$. $p_{2}$, ..., $p_{k}$
\item Distinct primes $e_{1}$. $e_{2}$, ..., $e_{k}$ such that \\
n = $(p_{1})^{(e_{1})}(p_{2})^{(e_{2})}(p_{3})^{(e_{3})} ... (p_{k})^{(e_{k})}$
\end{itemize}

and any other expression of n as a product of primes is identical to this, except perhaps for the order in which the terms are written

\subsubsection{Floor and Ceiling}

Given any x $\in \mathbb{R}$, the \underline{floor} of x, denoted $\lfloor$x$\rfloor$, is the unique integer n such that
n $\leq$ x $<$ n+1

Given any x $\in \mathbb{R}$, the \underline{ceilling} of x, denoted $\lceil$x$\rceil$, is the unique integer n such that
n - 1 $<$ x $\leq$ n+1

\subsubsection{The quotient-remainder theorem}

Given any integer n and a postiive integer d , there exists \underline{unique} integers q and r such that
\begin{itemize}
\item n = dq + r and Q $\leq$ r $<$ d
\end{itemize}

For integers \textit{a} and \textit{b}, and a positive integer \textit{d}, we say a is \underline{congruent} to b modulo d and write
\begin{itemize}
\item a $\equiv$ b (mod d)
\end{itemize}
if and only if d $|$ (a,b)

If a and b are not congruent modulo d, we write a /$\equiv$ b (mod d)
[don't know how to represent this]

\underline{Facts:}
\begin{itemize}
\item If n = dq + r, then n $\equiv$ r (mod d)
\item n $\equiv$ 0 (mod d) if and only if d$|$n
\end{itemize}

\subsection{Greatest common divisor}

For integers \textit{a,b} $\in \mathbb{Z}$, not both zero, the \underline{greatest common divisor} of a and b, denoted gcd(a,b), is the
integer \textit{d} which satisfies the following two properties:

\begin{itemize}
\item d $|$ a and d $|$ b
\item for all c $\in \mathbb{Z}$, if c $|$ a and c $|$ b then c $\leq$ d
\end{itemize}

Thus, d is the largest integer for which d $|$ a and d $|$ b

If gcd(a,b) = 1, then a and b have no common factors other than $\pm1$ and we call a and b coprime or relatively prime

What happens if a = b = 0?

For every d $\in \mathbb{Z}$, d satisfies d $|$ 0 (d $\neq$ 0) since 0 = d.0.
Thus, there is no greatest common divisor since there is no greatest integer.

However, if b $>$ 0, what is gcd(0,b)?
\begin{itemize}
\item d $|$ 0 for all d $\in \mathbb{Z}$ (d $\neq$ 0)
\item the greatest divisor of b is b
\end{itemize}

Thus, the greatest \underline{common} divisor is 1 \\

\underline{Fact:} If b $>$ 0 then gcd(0,b) = b \\

\underline{Fact:} If \textit{a} and \textit{b} are integers with b $\neq$ 0 and if \textit{q} and \textit{r} are integers such that:
\begin{align}
a = bq + r
\end{align}
then gcd(a,b) = gcd(b,r) \\

\underline{Why?}
\begin{itemize}
\item If d $|$ a and d $|$ b, then d $|$ bq so d $|$ (a - bq). Thus d $|$ r
\item If d $|$ r and d $|$ b, then d $|$ bq so d $|$ (bq+ r). Thus d $|$ a
\end{itemize}
So the common divisors of a and b are \underline{the same} as the common divisors of b and r.

\subsubsection{The Euclidean Algorithm}

To find gcd(a,b) where a,b $\in \mathbb{Z}$ and a $\geq$ b $>$ 0:
\begin{itemize}
\item Write a = bq + r, as in the quotient-remainder theorem
\item If r=0, then terminate gcd(a,b) = b
\item Otherwiste, replace (a,b) by (b,r) and \underline{repeat}
\end{itemize}

\underline{Example:} Find gcd(192, 132)
\begin{align}
a = 192, b = 132, 192 = 132.1 + 60, gcd(192,132) = gcd(132,60) \\
a = 132, b = 60, 132 = 60.2 + 12, gcd(132,60) = gcd(60,12) \\
a = 60, b = 12, 60 = 12.5 + 0, gcd(60,12) = gcd(12,0) \\
Therefore, gcd(192,132) = 12
\end{align}

\underline{Could this process repeat forever?}
No. By the quotient-remainder theorem; 0 $\leq$ r $<$ b. Since we use the old value of r as the new value of b when we repeat, this means that r becomes strictly smaller on each repetition.

Therefore, we must eventually reach r=0 and terminate.

\subsubsection{Lowest common multiple}

For nonzero itnegers a,b $\in \mathbb{Z}$, the \underline{lowest common multiple} of a and b is the smallest postive integer n for which a $|$ n and b $|$ n. We write this as lcm(a,b) \\

\underline{Fact:} Suppose a,b $\in \mathbb{Z}$ where a $\leq$ b $>$ 0. Then gcd(a,b).lcm(a,b) = ab

\subsection{Sequences}

A \underline{sequence} is an ordered list of elements. It can be finite or infinite.

Each individual element in a sequence is called a \underline{term}. We often denote the terms of sequences by lower case letters with subscripts.

\underline{Note:} The subscript of the initial term in a sequence does not need to be 1.

An \underline{explicit formula} or \underline{general formula} for a sequence is a rule showing how the value of a general term $a_{k}$ depends on k. \\

Different notations are used to denote such a sequence, such as:

\begin{itemize}
\item $(2^{k})_{k\leq0}$
\item $(2^{k})^{\infty}_{k\leq0}$
\item [figure out how to do this brace equivalents]
\end{itemize}

An \underline{alternating sequence} is a sequence in which the terms alternative between positive and negative \\

\underline{Example:} find a general formula for a sequence that has the following initial terms:

\begin{align}
2, \frac{3}{4}, \frac{4}{9}, \frac{5}{16}, \frac{6}{25}, \frac{7}{36}, ...
\end{align}

Let $a_{n}$ denote the general term and suppose the initial term is $a_{1}$. \\

Observe that the denominator in each term is a perfect and thus we can rewrite the given terms as:

\begin{align}
\frac{1+1}{1^{2}}, \frac{2+1}{2^{2}}, \frac{3+1}{3^{2}}, \frac{4+1}{4^{2}}, \frac{5+1}{5^{2}}, \frac{6+1}{6^{2}}, ...
\end{align}

$a_{n} = \frac{n+1}{n^{2}}$ \\

The sequence $(\frac{n+1}{n^{2}})$ has the given initial terms.

\subsubsection{Notation}

\underline{Summation Notation}

We use a Greek capital letter $\Sigma$ to indicate a sum. \\

If m, n $\in \mathbb{Z}$ and m $\leq$ n then
\begin{align}
\sum_{i=m}^{n} a_{i} = a_{m} + a_{m+1} + ... + a_{n}
\end{align}

m is the \underline{lower limit} of the summation
n is the \underline{upper limit} of the summation \\

If m=n then the sum only has one term \\

\underline{Dummy Variables}
The variable \textit{i} in $\sum_{i=m}^{n} a_{i}$ is a \textit{dummy variable}

One can use any letter here \textit{(as long as it is not already taken)} \\

\underline{Example:}

\begin{align}
\sum_{i=2}^{n} a_{i} = a_{2} + a_{3} + ... + a_{n} = \sum_{k=2}^{n} a_{k} = \sum_{j=2}^{n} a_{j}
\end{align}

The dummy variable is only relevant inside the su, which means you can reuse it outside the sum. \\

You can also perform a \textit{change of variables} \\

\underline{Example:}
\begin{align}
\sum_{i=2}^{6} (i-1)^{2} = 1^{2} + 2^{2} + 3^{2} + 4^{2} + 5^{2} \\
Let k = i-1 \\
When i = 2, k = 1 \\
When i = 6, k = 5 \\
\sum_{k=1}^{6} k^{2} = 1^{2} + 2^{2} + 3^{2} + 4^{2} + 5^{2} \\
\end{align}

\underline{Product Notation}
We use a Greek capital letter $\Pi$ \textit{(pi)} to indicate a product. \\

If m,n $\in \mathbb{Z}$ and m $\leq$ n then
\begin{align}
\prod_{i=m}^{n} a_{i} = a_{m} . a_{m+1} . ... . a_{n}
\end{align}

If m=n then the product only has one term

\subsubsection{Factorials}

For each positive integer n, we define n! \textit{(read n factorial)} to be
\begin{align}
n! = n(n-1)(n-2) ... 3.2.1 = \prod_{i=1}^{n} i
\end{align}

Also, we define 0! = 1

So,
\begin{align}
0! = 1 \\
1! = 1 \\
2! = 2.1 = 2 \\
3! = 3.2.1 = 6 \\
4! = 4.3.2.1 = 24
\end{align}

\subsubsection{Properties of Summation and Product}

If $a_{m}$, $a_{m+1}$, ... and $b_{m}$, $b_{m+1}$, ... are sequences of real numbers, and c is any real number, then for any integer n $\geq$ m, the following hold:
\begin{enumerate}
\item $\sum_{i=m}^{n} a_{i} \pm \sum_{i=m}^{n} b_{i} =  \sum_{i=m}^{n} (a_{i} \pm b_{i})$ \\
\textit{(adding/subtracting over the same range)}

\item $\sum_{i=m}^{n} ca_{i} = c\sum_{i=m}^{n} a_{i}$ \\
\textit{(taking out a common factor)}

\item $(\prod_{i=m}^{n} a_{i}) (\prod_{i=m}^{n} b_{i}) = \prod_{i=m}^{n} a_{i}.b_{i}$
\end{enumerate}

Suppose P(n) is a predicate and we know:
\begin{itemize}
\item P(1)
\item P(1) $\rightarrow$ P(2)
\item P(2) $\rightarrow$ P(3)
\item P(3) $\rightarrow$ P(4)
\end{itemize}

Can we conclude that P(4) is true? Yes. By repeated use of modus ponens.

\subsection{Mathematical Induction}

\subsubsection{The principle of mathematical induction}

Let P(n) be a predicat that is defined for every integer n $\geq$ a, where a is some fixed integer. \\

Suppose
\begin{enumerate}
\item P(a) is true
\item For every integer k $\geq$ a, P(k) $\rightarrow$ P(k+1)
\end{enumerate}

How to use the principle of mathematical induction: \\

To prove that: For every integer n $\geq$ a, P(a).

\begin{enumerate}
\item Basis Step: Prove P(a)
\item Inductive Step: Prove that: For every integer k $\geq$ a, P(k) $\rightarrow$ P(k+1)
\begin{itemize}
\item Suppose k is an integer, where k $\geq$ a, and P(k) is true. \textit{(this is called the \textbf{Inductive Hypothesis}}
\item Using this, show that P(k+1) is true
\end{itemize}
\end{enumerate}

\subsubsection{The principle of strong mathematical induction}

Let P(n) be a predicate that is defined for every integer n $\geq$ a, where a is some fixed integer, and let b be an integer where b $\geq$ a \\

Suppose

\begin{enumerate}
\item Basis Step: P(a), P(a+1), ..., P(b) are all true
\item Inductive Step: For every integer k $\geq$ b, if P(a), P(a+1), ..., P(k) are all true, then P(k+1) is true
\end{enumerate}

Then P(n) is true for every integer n $\geq$ a \\

It can be shown that this is \textbf{equivalent}  to the ordinary principle of mathematical induction \\

\underline{Example:} Prove that
For every integer n $\geq$ 8, we can form n cent postage using 3 cent and/or 5 cent stamps \\

Before starting a proof, observe that:
\begin{align}
8 = 5 + 3 \rightarrow 11 = 5 + 3 + 3 \\
9 = 3 + 3 + 3 \rightarrow 12 = 3 + 3 + 3 + 3 \\
10 = 5 + 5 \rightarrow 13 = 5 + 5 + 3 \\
\end{align}

We now use this idea in a formal proof.

\underline{Proof:} Let P(n) be the predicate
"n cent postage can be formed using only 3 cent and/or 5 cent stamps" \\

Basis Step: We can prove P(8), P(9) and P(10) directly, since:
\begin{align}
8 = 5 + 3 \\
9 = 3 + 3 + 3 \\
10 = 5 + 5
\end{align}

Inductive Hypothesis: Suppose that for some integer k $\geq$ 10, P(8), ..., P(k) are all true. We will use this to prove P(k+1) 

Since k $\geq$ 10, we have k-2 $\geq$ 8. Thus, by the Inductive Hypothesis, we can form (k-2) cents using 3 cent and/or 5 cent stamps.

Now we can add one more 3 cent stamp to make (k+1) postage and so P(k+1) is true.

Therefore, by strong induction, it follows that P(n) is true for every integer n $\geq$ 8

\subsubsection{The Well-Ordering principle for the integers:}

If S is a non-empty set of integers all of which are greater than some fixed integer, then S has a least element \\

It can be shown that the well-ordering principle is equivalent to the principle of mathematical induction, or even strong mathematical induction. All three are equivalent. \\

For each of the following sets, do they have a least element?
\begin{itemize}
\item The set of all integers greater than 2? Yes
\item The set of all natural numbers $\mathbb{N}$? Yes
\item The set of all odd integers? No
\item The set of all primes? Yes
\item The set of positive real numbers? No
\end{itemize}

\subsubsection{Recurrence}

A \underline{recurrence relation} for a sequence $a_{0},a_{1},a_{2},...$ is a formula that relates each term $a_{k}$ to some of its predecessors $a_{k-1}, ..., a_{k-i}$ where i $\in \mathbb{Z}$ and k - i $\geq$ 0 

The \underline{initial conditions} for such a recurrence relation specify the values of some of the initial terms

\underline{Example:} The fibonacci sequence is defined recursively by
\begin{align}
F_{0} = 1, F_{1} = 1 (initial conditions) and F_{n} = F_{n-1} + F_{n-2} (recurrence relation) for n \geq 2
\end{align}

\subsubsection{Forms of defining a sequence}

\underline{Ways to define sequences}

A sequence can be defined
\begin{itemize}
\item Informally, by listing the first few terms of the sequence until the pattern becomes obvious
\begin{itemize}
\item \underline{Ex:} 1, 1, 2, 6, 24, 120, 720, ...
\end{itemize}

\item With a general formula, by stating how a term $a_{n}$ depends on n and stating where it starts
\begin{itemize}
\item \underline{Ex:} $(n!)_{n\geq0}$
\end{itemize}

\item Recursively, by giving a recurrence relation relating later terms in the sequence to earlier ones and also some initial conditions
\begin{itemize}
\item \underline{Ex:} $a_{0}$ = 1, $a_{n}$ = n.$a_{n-1}$ For n $\geq$ 1
\end{itemize}
\end{itemize}

\underline{Example:} Let $c_{0}$, $c_{1}$, $c_{2}$, ... be a sequence defined by
\begin{align}
c_{0} = 1, c_{1} = 2, c_{2} = 3 \\
c_{n} = 3c_{n-1} - c_{n-2} - c_{n-3} for n \geq 3 
\end{align}

Write out the first 6 terms for the sequence
\begin{align}
c_{0} = 1 \\
c_{1} = 2 \\
c_{2} = 3 \\
c_{3} = 3c_{2} - c_{1} - c_{0} = 3.3 -2 -1 = 6 \\
c_{4} = 3c_{3} - c_{2} - c_{1} = 3.6 -3 -2 = 13 \\
c_{5} = 3c_{4} - c_{3} - c_{2} = 3.13 -6 -3 = 30 \\
\end{align}

\underline{Example:} Show that the sequence $a_{k} = 3 - 2^{k}$, for k $\geq$ 0, 
satisfies the recurrence relation $a_{n}$ = $2a_{n-1}$ for n $\geq$ 1 \\

The sequence is $(3.2^{k})_{k\geq0}$ = 3, 6, 12, 24, 48, ...

For every integer n $\geq$ 1 we have $a_{n} = 3 - 2^{n}$ and $a_{n-1} = 3 - 2^{n-1}$ 

Hence 
\begin{align}
a_{n} = 3 - 2^{n} \\
= 3.2.2^{n-1} \\
= 2.(3.2^{n-1}) \\
= 2.a_{n-1}
\end{align}

Therefore, the sequence $(3.2^{k})_{k\geq0}$ satisfies the recurrence relation $a_{n}$ = $2a_{n-1}$ for n $\geq$ 1 \\

We have seen that sequences of numbers can be defined recursively. Many other objects can be defined recursively as well, such as: sets, sums, products and functions ...

A recursive definition for a set of objects requires three things:
\begin{enumerate}
\item \textbf{BASE:} A statement that a certain object belongs in a set
\item \textbf{RECURSION:} A collection of rules showing how to form new objects for the set from existing ones in the set
\item \textbf{RESTRICTION:} A statement that no objects belong to the set other than those arising from steps \textbf{1} and \textbf{2}
\end{enumerate}

Given a sequence that is defined recursively, one may wish to find an explicit formula for the sequence \\

\underline{Method of Iteration} \\

\begin{enumerate}
\item Use iteration to write down several terms of the sequence and \textbf{guess} an explicit formula
\item Use induction to prove that the guess is correct
\end{enumerate}

\underline{Example:} Find an explicit formula for the sequence defined by 
\begin{align}
b_{0} = 2 \\
b_{n} = b_{n-1} + 5 for n \geq 1
\end{align}

We have $b_{0} = 2$
\begin{align}
b_{1} = b_{0} + 5 = 2 + 5 = 7 \\
b_{2} = b_{1} + 5 = 7 + 5 = 12 \\
b_{3} = b_{2} + 5 = 12 + 5 = 17 \\
b_{4} = b_{3} + 5 = 17 + 5 = 22 \\
\end{align}

Each term is 5 more than the previous term. Note that repeated addition can be written as multiplication \\

Guess: $b_{n}$ = 5n + 2 for n $\geq$ 0 \\

Next, we prove that for n $\geq$ 0, $b_{n}$ = 5n + 2 \\

\underline{Proof}\\

Basis Step: When n = 0 $\rightarrow$ $b_{0}$ = 2 and 5(0) + 2 \\

Inductive Hypothesis: Suppose from some integer k $\geq$ 0, $b_{k}$ = 5k + 2\\

Now $b_{k+1}$ = $b_{k}$ + 5 by definition (since k + 1 $\geq$ 1) \\
= 5k + 2 + 5 by the Inductive Hypothesis \\
= 5(k+1) + 2 \\

Therefore, by the principle of mathematical induction, for every integer n $\geq$ 0, $b_{n}$ = 5n + 2 \\

\subsubsection{Arthimetic Sequences}

A sequence $a_{0}$, $a_{1}$, $a_{2}$, ... is an \underline{arthimetic sequence} if and only if there is a constant \textit{d}
such that $a_{k}$ - $a_{k-1}$ + d for every k $\geq$ 1 \\

It follows that an explicit for the sequence is
\begin{align}
a_{n} = a_{0} + dn
\end{align}
for every integer n $\geq$ 0 \\

\subsubsection{Geometric Sequences}

A sequence $a_{0}$, $a_{1}$, $a_{2}$, ... is a \underline{geometric sequence} if and only if there is a constant \textit{r} such that 
$a_{k}$ = r$a_{k-1}$ for every integer k $\geq$ 1 \\

It follows that an explicit formula for the sequence is $a_{n}$ = $a_{0}.r^{n}$ for every n $\geq$ 0 \\

Note: It is not always possible to guess an explicit formula, and, in fact, some recursively defined sequences do not have
an explicit formula \\ 

\underline{Example:} $F_{0}$ = 1, $F_{1}$ = 1, $F_{n}$ = $F_{n-1}$ + $F_{n-2}$ for n $\geq$ 2 \\

The explicit formula is
\begin{align}
F_{n} = \frac{1}{\sqrt{5}} ( \frac{1+\sqrt{5}}{2} )^{n+1} - \frac{1}{\sqrt{5}} ( \frac{1-\sqrt{5}}{2} )^{n+1}
\end{align}

\section{Set theory}

A \underline{set} S is a collection of objects, which are called the \underline{elements} of S \\

If x is in S, one writes x $\in$ S \\

If not, one writes x $\notin$ S \\

We can sometimes list the elements of S with curly braces: S = \{$x_{1}$, $x_{2}$, ...\} \\

\underline{Example:} S = \{2, 3, 4\} \\

2 $\in$ S but 1 $\notin$ S and $\pi \notin$ S \\

\underline{Example:} Let E be the set of positive even integers, so E = \{2, 4, 6, 8, ...\}, which is an infinite set. 20 $\in$ E but
21 $\notin$ E and -4 $\notin$ E \\

Order does not matter. \{2, 3, 5, 8\} = \{3, 8, 5, 2\} \\

Repetitions are ignored \{1, 3, 4\} = \{1, 1, 3, 4\} \\
= \{4, 3, 1, 3, 4, 1, 1\} \\

One can define a set by a property that its elements must satisfy:
\begin{align}
A = \{x \in S | P(x)\}
\end{align}
means that the elements of A are precisely those elements of S for which the predicate P(x) is true. \\

\underline{Example:} The set of all even integer is 
\begin{align}
\{n \in \mathbb{Z} | n = 2k for some k \in \mathbb{Z}\}
\end{align}

\underline{Example:}
\begin{align}
\{x \in \mathbb{Z} | 3 < x < 7\} = \{4, 5, 6\}
\end{align}

The elements of a set can be sets themselves \\ 
\underline{Example: }
\begin{align}
A=\{1, 2, \{3\}, \{5,6\}\}
\end{align}

Here 2 $\in$ A and \{3\} $\in$ A \\
but \{2\} $\notin$ A and 5 $\notin$ A \\

\subsection{Subsets}

If A and B are sets, A is called a \underline{subset} of B, written A 

\section{Logical Equivalences}

Given any statement variables '\textit{p}', '\textit{q}' and '\textit{r}', a tautology '\textbf{t}' and contradiction '\textbf{c}', the following
logical equivalences hold.


\subsection{Commutative laws}
\begin{itemize} 
\item \textit{p} $\wedge$ \textit{q} $\equiv$ \textit{q} $\wedge$ \textit{p}
\item \textit{p} $\lor$ \textit{q} $\equiv$ \textit{q} $\lor$ \textit{p}
\end{itemize}

\subsection{Associative laws}
\begin{itemize}
\item (\textit{p} $\wedge$ \textit{q}) $\wedge$ \textit{r} $\equiv$ \textit{p} $\wedge$ (\textit{q} $\wedge$ \textit{r})
\item (\textit{p} $\lor$ \textit{q}) $\lor$ \textit{r} $\equiv$ \textit{p} $\lor$ (\textit{q} $\lor$ \textit{r})
\end{itemize}

\subsection{Distributive laws}
\begin{itemize}
\item \textit{p} $\wedge$ (\textit{q} $\lor$ \textit{r}) $\equiv$ (\textit{p} $\wedge$ \textit{q}) $\lor$ (\textit{p} $\wedge$ \textit{r})
\item \textit{p} $\lor$ (\textit{q} $\wedge$ \textit{r}) $\equiv$ (\textit{p} $\lor$ \textit{q}) $\wedge$ (\textit{p} $\lor$ \textit{r})
\end{itemize}

\subsection{Identity laws}
\begin{itemize}
\item \textit{p} $\wedge$ \textbf{t} $\equiv$ \textit{p}
\item \textit{p} $\lor$ \textbf{c} $\equiv$ \textit{p}
\end{itemize}

\subsection{Negation laws}
\begin{itemize}
\item \textit{p} $\lor$ $\sim$\textit{p} $\equiv$ \textbf{t}
\item \textit{p} $\wedge$ $\sim$\textit{p} $\equiv$ \textbf{c}
\end{itemize}

\subsection{Double negative laws}
\begin{itemize}
\item $\sim$($\sim$\textit{p}) $\equiv$ \textit{p}
\end{itemize}

\subsection{Idempotent laws}
\begin{itemize}
\item \textit{p} $\lor$ \textit{p} $\equiv$ \textit{p}
\item \textit{p} $\wedge$ \textit{p} $\equiv$ \textit{p}
\end{itemize}

\subsection{Universal bound laws}
\begin{itemize}
\item \textit{p} $\lor$ \textbf{t} $\equiv$ \textbf{t}
\item \textit{p} $\wedge$ \textbf{c} $\equiv$ \textbf{c}
\end{itemize}

\subsection{De Morgan's laws}
\begin{itemize}
\item $\sim$(\textit{p} $\wedge$ \textit{q}) $\equiv$ $\sim$\textit{p} $\lor$ $\sim$\textit{q}
\item $\sim$(\textit{p} $\lor$ \textit{q}) $\equiv$ $\sim$\textit{p} $\wedge$ $\sim$\textit{q}
\end{itemize}

\subsection{Absorption laws}
\begin{itemize}
\item \textit{p} $\lor$ (\textit{p} $\wedge$ \textit{q}) $\equiv$ \textit{p}
\item \textit{p} $\wedge$ (\textit{p} $\lor$ \textit{q}) $\equiv$ \textit{p}
\end{itemize}

\subsection{Negations of t and c}
\begin{itemize}
\item $\sim$\textbf{t} $\equiv$ \textbf{c}
\item $\sim$\textbf{c} $\equiv$ \textbf{t}
\end{itemize}


\end{document}