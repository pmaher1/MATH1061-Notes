\documentclass{article}

\usepackage{amsmath}
\usepackage{amssymb}

\title{MATH1061 Assignment 1}
\author{Paddy Maher}

\begin{document}
\maketitle

\section{}
$(( q \rightarrow p ) \lor r ) \lor ( \sim r \wedge p )$
Use a truth table to determine truth values.

\begin{tabular} { c|c|c|c c c c c }
p & q & r & $(( q \rightarrow p )$ &  $\lor r )$&  $\lor$ &$( \sim r$& $\wedge p )$ \\
\hline
T & T & T & T & T & T & F & F \\
T & T & F & T & T & T & T & T \\
T & F & T & F & T & T & F & F \\
T & F & F & F & F & T & T & T \\
F & T & T & F & T & T & F & F \\
F & T & F & F & F & F & T & F \\
F & F & T & T & T & T & F & F \\
F & F & F & T & F & T & T & F \\
\end{tabular}

Take notice to the column involving the \textbf{OR} ($\wedge$, \textit{3rd column from the right
and 5th column from the left}). Notice in the one case of which it is false; p is True, q is False and r is False.

Thus, the final values are p = True, q = False and r = False.

\section{}

Begin with the statement $\sim (( p \wedge ( q \rightarrow \sim r )) \wedge ( r \rightarrow ( p \lor \sim q )))$

Recognise $p \rightarrow q \equiv \sim p \lor q$

Therefore, the original statment may be simplified to:

\begin{align*}
& \sim (( p \wedge ( \sim q \wedge \sim r )) \wedge (  \sim r \wedge ( p \wedge \sim q ))) \\
& \equiv \sim ((( p \wedge \sim q ) \wedge \sim r )) \wedge (  \sim r \wedge ( p \wedge \sim q )))    \textit{Associative law} \\
& \equiv \sim (( p \wedge \sim q ) \wedge \sim r )    \textit{Idempotent law}  \\
& \equiv \sim p \wedge q \wedge r    \textit{De Morgan's law} \\
& \equiv ( \sim p \wedge q ) \wedge r    \textit{Associative law}
\end{align*}

$\therefore \sim (( p \wedge ( q \rightarrow \sim r )) \wedge ( r \rightarrow ( p \lor \sim q ))) \equiv ( \sim p \wedge q ) \wedge r $

\section{}
Of the form of Modus Ponens \\
Students of MATH1061 hand in their maths assignments. \\
There are students of MATH1061 \\
Therefore, maths assignments shall be submitted.

\section{}

\begin{align*}
& q \rightarrow ( p \wedge n ) \\
& s \rightarrow p \\
& n \lor ( s \wedge r ) \\
& r \\
& \therefore r \rightarrow p \\
\end{align*}

This is an \underline{invalid} argument \\

\underline{Explanation} \\
Assume all predicates are true. It is possible that all predicates can be true and the conclusion can be false.
For example, 'r' is true, 'n' is true (thus making $n \lor ( s \wedge r )$ true) and the two conditional statements being true; the conclusion
can still be false.

\section{}

\begin{enumerate}
\item[a)] $\exists n \in \mathbb{Z}$ such that $n^2 > 100$ \\
True, this is for true for all $|n| > 10$
\item[b)] $\forall n \in \mathbb{Z}, n|5 \rightarrow n|10$ \\
False, take n = 5 for a counterexample. \textit{(5 is divisible by 5 whereas 5 is not divisible by 10)}
\item[c)] $\forall  x \in \mathbb{R}, \sqrt{x} \notin \mathbb{Q} \rightarrow x \in \mathbb{Q}$ \\
Not true, take x = 4
\end{enumerate}

\section{}

\begin{enumerate}
\item[a)] $\exists a \in \mathbb{Z}$ such that $\forall b \in \mathbb{Z},$ ab $>$ 0 \\

Negation: $\forall a \in \mathbb{Z}, \exists b \in \mathbb{Z}$ such that ab $\leq$ 0 \\

The statement isn't true, however the negation is true

The statement can be disproven through analysing 'a' as either negative, positive or 0. It cannot be 0 and hold the inequality,
if it is of any sign, 'b' of the opposite sign will ensure that the product will be negative, thus ab < 0, negating the previous equality.
Or this could be more simply analysed as it doesn't account when 'b' is equal to 0. As the product \textit{(ab)} is 0. \\

The negation is true, consider 'b' is equal to 0. Therefore, there is a case where ab $\leq$ 0

\item[b)]
$\exists a, b \in \mathbb{Z}^{+}$ such that $\forall c, d \in \mathbb{Z}^{+}, \frac{a}{b} = \frac{c}{d}$ \\

Negation: $\forall a, b \in \mathbb{Z}^{+}$ such that $\exists c, d \in \mathbb{Z}^{+}, \frac{a}{b} \ne \frac{c}{d}$ \\

As in \textbf{(a)}, the statement isn't true, however the negation is true

Note that in the original statement, 'b' cannot equal 0. In this case, 'c' must be a multiple of 'a' as 'd' must be a multiple of 'b' in order
for this equality to hold. However, the variables are not binded by such conditions and contains all
integers, therefore, this statement isn't true.

The negation is true, due to being the negation of the original statement which was false.

\end{enumerate}

\end{document}