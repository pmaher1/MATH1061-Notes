\documentclass{article}

\usepackage{amsmath}
\usepackage{amssymb}
\usepackage{graphicx}
\title{MATH1061 Course Notes}
\author{Paddy Maher}

\begin{document}
\maketitle
\newpage

\section{Logic}
\subsection{Logical Connectives}
\subsubsection{Basic logical connectives}
For a given logical statement come logical connectives. Basic logical connectives include:
\begin{itemize}
\item \textbf{not} = $\sim$
\item \textbf{and} = $\wedge$
\item \textbf{or} = $\lor$
\item \textbf{exclusive or} = $\oplus$
\end{itemize}

\subsubsection{Logical Equivalence}
Given two statement forms, you can show that they are logically equivalent by using a truth table or by using 
the laws of logical equivalence. \\
The logical equivalence between two statements is demonstrated by the symbol $\equiv$

[insert truth table]

\subsubsection{Conditional logical connectives}
Logical connectives and equivalences, for given truth statements '\textit{p}' and '\textit{q}'
\begin{itemize}
\item \textbf{if .. then} = $\rightarrow$
\item \textbf{if and only if} = $\leftrightarrow$
\item \textit{p} $\rightarrow$ \textit{q} $\equiv$ 
$\sim$ \textit{p} $\lor$ \textit{q} $\equiv$ 
$\sim$ \textit{q} $\rightarrow$ $\sim$ \textit{p}
\end{itemize}

\subsubsection{Order of Operations}
\begin{enumerate}
\item $\sim$
\item $\wedge$ and $\lor$, use parentheses to specify. If no parentheses given, work from left to right.
\item  $\rightarrow$ and $\leftrightarrow$, use parentheses to specify. If no parentheses given, work from left to right.
\end{enumerate}

\subsection{Necessary and sufficient conditions}
For given truth statements '\textit{p}' and '\textit{q}':
\begin{itemize}
\item \textit{p} is a necessary condition for \textit{q} means "if $\sim$\textit{p} then $\sim$\textit{q}" or
equivalently "if \textit{q} then \textit{p}" or "\textit{q} only if \textit{p}".
\item \textit{p} is a sufficient condition for \textit{q} means "if \textit{p} then \textit{q}" or equivalently
"\textit{q} if \textit{p}"
\end{itemize}

\subsection{Definitions}
\subsubsection{Tautology and contradictions}
\begin{itemize}
\item A \underline{tautology} is a statement form which always takes truth values "\textbf{true}" for all possible
truth values of its variables.
\item A \underline{contradiction} is a statement form which always takes truth values "\textbf{false}" for all possible
truth values of its variables.
\end{itemize}

\subsubsection{Contrapositive}
For given truth statements '\textit{p}' and '\textit{q}' \\
The \underline{contrapositive} of \textit{p} $\rightarrow$ \textit{q} is $\sim$\textit{q} $\rightarrow$ $\sim$\textit{p}
\begin{itemize}
\item These are logically equivalent: \\
\textit{p} $\rightarrow$ \textit{q} $\equiv$  $\sim$\textit{q} $\rightarrow$ $\sim$\textit{p}
\end{itemize}

\subsubsection{Biconditional}
For given truth statements '\textit{p}' and '\textit{q}' \\
The \underline{biconditional} of  \textit{p} and \textit{q}, denoted \textit{p} $\leftrightarrow$ \textit{q},
 is defined by the following truth table:

[insert truth table]

\subsection{Arguments}

\subsubsection{Premises}

Given a collection of statements '$p_1 , p_2, ... , p_n$' (called \textbf{premises}) and another statement 'q' (called the conclusion),
an '\textit{argument}' is the assertion that the conjunction of the premisees implies the conclusion.

\begin{align*}
& p_1 \\ 
& p_2 \\
& ... \\
& p_n \\
& \therefore q
\end{align*}

\subsubsection{Arguments; validity and invalidity}

\underline{Definition; valid argument}
An argument is \textbf{valid} if whenever all of the premises are true, the conclusion is also true. 

Thus, an argument is valid if ($p_1 \wedge p_2 \wedge ... \wedge p_n \rightarrow q$) is a tautology.

\underline{Definition; invalid argument}
An argument is \textbf{invalid} if it is possible to have a situation in which all of the premises are true but the conclusion is false.

We can check whether an argument is valid or invalid using a truth table.

\subsubsection{Rules of Inference}

Modus Ponens 
\begin{align*}
& p \rightarrow q \\ 
& p \\
& \therefore q
\end{align*}

Modus Tollens
\begin{align*}
& p \rightarrow q \\ 
& \sim q \\
& \therefore \sim p
\end{align*} \\

Generalisation
\begin{align*}
& p \\ 
& \therefore p \lor q \\ \\
& q \\ 
& \therefore p \lor q 
\end{align*} \\

Specialisation
\begin{align*}
& p \wedge q \\ 
& \therefore p \\ \\
& p \wedge q \\ 
& \therefore q 
\end{align*} \\

Conjunction
\begin{align*}
& p \\ 
& q \\
& \therefore p \wedge q
\end{align*} \\

Elimination
\begin{align*}
& p \lor q \\ 
& \sim q \\
& \therefore p \\ \\
& p \lor q \\ 
& \sim p \\
& \therefore q
\end{align*} \\

Transitivity
\begin{align*}
& p \rightarrow q \\ 
& q \rightarrow r \\
& \therefore p \rightarrow r
\end{align*} \\

Proof by Division into cases
\begin{align*}
& p \lor q \\ 
& p \rightarrow q \\
& q \rightarrow r \\
& \therefore r
\end{align*} \\

Contradiction Rule
\begin{align*}
& \sim p \rightarrow (contradiction) \\
& \therefore p
\end{align*} \\

\subsubsection{Alternative method for determining validity}

If an argument is \textit{'invalid'} then there is a situation where all the premises are true but the conclusion is false.

Attempt to see whether this is possible. To do this, look for truth values which make all premises true yet the conclusion is false.

If such truth values can be found, then the argument is \textit{invalid} \\

Summary of this method:
\begin{itemize}
\item Try to make all the premises true and the conclusion false
\item If this can be done, then the argument is \underline{invalid}
\item On the other hand, if this is \textbf{impossible} to do, then the argument is \underline{valid}
\end{itemize}

\underline{Checks for Validity}
\begin{itemize}
\item Use a truth table
\item Use rules of inference
\item Attempt to find turht values that make all premises true but the conclusion false.
\end{itemize}

\subsubsection{Predicates and domains}

A \underline{predicate} is a sentence that contains finitely many variables, and which becomes a statement if the variables are given
specific values. \\

The \underline{domain} of each variable in a predicate is the set of all possible values that may be assigned to it. \\

The \underline{truth set} of a predicate P(x) is the set of all values in the domain that, when assigned to x, make P(x) a true statement.

\underline{Common Domains}
\begin{itemize}
\item Integers: $\mathbb{Z}$ = [ ..., -3, -2, -1, 0, 1, 2, 3, .. ]
\item Positive integers: $\mathbb{Z}^{+}$ = [ 1, 2, 3, ... ]
\item Non-negative integers: $\mathbb{Z}^{non-neg}$ = [ 0, 1, 2, 3, ... ]
\item Natural numbers: $\mathbb{N}$ = [ 1, 2, 3, ... ]
\item Rational numbers: $\mathbb{Q}$ = [ $\frac{a}{b}$ $|$ a, b $\in \mathbb{Z} \wedge$ b $\neq$ 0 ]
\item Real numbers: $\mathbb{R}$ = entire number line
\end{itemize}

\subsection{Quantifiers}

\subsubsection{Universal and Existental quantifiers}

\underline{The Universal quantifier}

The symbol '$\forall$' denotes \textit{"for all"} and is called the \textbf{universal quantifier}

Let Q(x) be a predicate and \textbf{D} be the domain of x. The \textbf{universal statement}

$\forall$ x $\in$ \textbf{D}, Q(x)

is true if and only if Q(x) is true for every x in \textbf{D}. It is false if and only if Q(x) is false for at least one x in \textbf{D} \\

\underline{The Existental Quantifier}

The symbol '$\exists$' denotes "there exists" and is called the \textbf{existential quantifier}

Let Q(x) be a predicate and \textbf{D} be the domain of x. The \textbf{existential statement}

$\exists$ x $\in$ \textbf{D} such that Q(x)

is true if and only if Q(x) is true for at least one x in \textbf{D}. It is false if and only if Q(x) is false for every x in \textbf{D}

\subsubsection{Negation of Quantified Statements}

\underline{Universal Statement:}

$\forall$ x $\in$ \textbf{D}, Q(x)

The negation of this statement is logically equivalent to:

$\exists$ x $\in$ \textbf{D} such that $\sim$ Q(x) \\

\underline{Existential Statement:}

$\exists$ x $\in$ \textbf{D} such that Q(x) 

The negation of this statement is logically equivalent to:

$\forall$ x $\in$ \textbf{D}, $\sim$ Q(x) \\

\underline{Universal Conditional Statement}

$\forall$ x $\in$ \textbf{D} if P(x) then Q(x)

The negation of this statment is logically equivalent to:

$\exists$ x $\in$ \textbf{D} such that $\sim$ if P(x) then Q(x) 

which is;

$\exists$ x $\in$ \textbf{D} such that P(x) $\wedge \sim$Q(x)

\subsection{Multiple quantifiers}

\subsubsection{Intro to multiple quantifers}

The predicate x $\leq$ y for real numbers \textit{x} and \textit{y} involves more than one variable.

Notation such as P(x,y) is used to denote such predicates.

Such predicates often appear in statments that involve more than one quantifier \\

In order to establish the truth of a statment of the form:

$\forall$ x $\in$ \textbf{D}, $\exists$ y $\in$ \textbf{D} such that P(x,y)

One must allow another to pick whatever element x $\in$ \textbf{D} they wish, and then must preceed with finding an element
y $\in$ \textbf{E} which makes P(x,y) true. \\

In order to establish the truth of a statment of the form:

$\exists$ x $\in$ \textbf{D} such that $\forall$ y $\in$ \textbf{D}, P(x,y)

One must find one particular x $\in$ \textbf{D} which makes P(x,y) true no matter which y $\in$ \textbf{D} might be chosen
for you.

\subsubsection{Negation of statements with multiple quantifers}

The statement: 

$\forall$ x $\in$ \textbf{D}, $\exists$ y $\in$ \textbf{E} such that P(x,y)

Negates to: 

$\exists$ x $\in$ \textbf{D} such that $\sim$ ($\exists$ y $\in$ \textbf{E} such that P(x,y))

Which is:

$\exists$ x $\in$ \textbf{D} such that $\forall$ y $\in$ \textbf{E} $\sim$ P(x,y) \\


The statement: 

$\exists$ x $\in$ \textbf{D} such that $\forall$ y $\in$ \textbf{E}, P(x,y)

Negates to:

$\forall$ x $\in$ \textbf{D}, $\sim$ ($\forall$ y $\in$ \textbf{E}, P(x,y))

Which is:

$\forall$ x $\in$ \textbf{D}, $\exists$ y $\in$ \textbf{E} such that $\sim$ P(x,y))

\section{Proofs and Number Theory}

\subsection{Proofs}

\subsubsection{Even and Odd}

An integer 'n' is \underline{even} if and only if 'n' is twice some integer.
\begin{itemize}
\item That is: n is even $\leftrightarrow \exists k \in \mathbb{Z}$ such that n = 2k
\end{itemize}

An integer 'n' is \underline{odd} if and only if 'n' is twice some integer.
\begin{itemize}
\item That is: n is odd $\leftrightarrow \exists k \in \mathbb{Z}$ such that n = 2k + 1
\end{itemize}

\subsubsection{Prime and Composite}

An integer 'n' is \underline{prime} if and only if n $>$ 1 and for all positive integers 'r' and 's', if n = rs, then r = 1 or s = 1.

An integer 'n' is \underline{composite} if and only if n $>$ 1 and n = rs for some positive integers 'r' and 's' with r $\neq$ 1 and s $\neq$ 1

\underline{In symbols:}

\begin{itemize}
\item n is prime $\leftrightarrow$ n $>$ 1 $\wedge \forall$ r,s $\in \mathbb{Z}^{+}$, n = rs $\rightarrow$ ( r=1 $\wedge$ s=1 )
\item n is composite $\leftrightarrow$ n $>$ 1 $\wedge \exists$ r, s $\in \mathbb{Z}^{+}$ such that n = rs $\wedge$ r $\neq$ 1 $\wedge$ s $\neq$ 1
\item Note: 1 is neither prime nor composite.
\end{itemize}

\subsubsection{Direct proofs}

\underline{Proving Existental Statements}

To show $\exists$ x $\in$ \textbf{D} such that P(x) is true, it is enough to find one example of an element x in \textbf{D} for which P(x) is true. \\

\underline{Direct Proof of Universal Statements}

One way to show that $\forall$ x $\in$ \textbf{D}, P(x) is true is by a \underline{direct proof}.

\begin{enumerate}
\item Suppose x $\in$ \textbf{D}
\item Show that P(x) is true
\end{enumerate}

Method of direct proof to show that $\forall$ x $\in$ \textbf{D} if P(x) then Q(x) is true:
\begin{enumerate}
\item Suppose x $\in$ \textbf{D} and P(x) is true
\item Show that the conclusion Q(x) is true using definitions, previously established results, and the rules for logical inference.
\end{enumerate}

\underline{How to write a proof}

\begin{itemize}
\item Write the theorem to be proved
\item Clearly mark the beginning of the proof with the word \textit{"Proof"}
\item Use precise definitions for any mathematical terms
\item Write the proof in complete sentences
\item Give a reason for each assertion in your proof
\item Make your proof self-contained
\item Display equations and inequalities clearly
\item Conclude by stating what it is you have proved
\end{itemize}

\subsubsection{Other forms of proof}

\underline{Disproof by Counterexample}
To show that $\forall$ x $\in$ \textbf{D} if P(x) then Q(x) is false, find a value of x $\in$ \textbf{D} for which P(x) is true and Q(x) is false. \\ \\ 

\underline{Method of Proof by Contradiction}

\begin{enumerate}
\item Assume that the statement is false
\item Show that this leads logically to a contradiction
\begin{itemize}
\item So it is impossible that the statement is false
\end{itemize}
\item Conclude that the statement is true
\end{enumerate}

\underline{Method of Proof by Contradiction to show that $\forall$ x $\in$ \textbf{D}, if P(x) then Q(x) is true:}

\begin{enumerate}
\item Assume that the statement to be proved is false. Thus, the \textit{negation} of the statement is true: $\exists$ x $\in$ \textbf{D} such that P(x) and $\sim$ Q(x)
\item Show that this assumption leads logically to a contradiction
\item Conclude that the statement to be proved is true.
\end{enumerate}

\underline{Method of Proof by Contraposition}

\begin{enumerate}
\item Express the statement to be proved in the form: $\forall$ x $\in$ \textbf{D}, if P(x) then Q(x)
\item Rewrite this statement in the \underline{contrapositive} form: $\forall$ x $\in$ \textbf{D}, if $\sim$ Q(x) the $\sim$ P(x)
\item Prove the contrapositive by direct proof:
\begin{itemize}
\item Suppose x $\in$ \textbf{D} and Q(x) is false
\item Show that P(x) is false
\end{itemize}
\end{enumerate}

\subsection{Types of numbers}

\subsubsection{Rational numbers}

A real number is \underline{rational} if and only if it can be expressed as a quotient of two integers with a non-zero denominator.

The set of all rational numbers is denoted by $\mathbb{Q}$

A real number that is not rational is \underline{irrational}

r is rational $\leftrightarrow$ $\exists$ a, b $\in \mathbb{Z}$ such that r = $\frac{a}{b}$ and b $\neq$ 0

\underline{Fact:}
The decimal expansion of a rational number either repeats or terminates

\underline{Fact:}
The decimal expansion of an irrational number does not repeat and does not terminate

\underline{Theorem:}
For all rational numbers r and s where r $<$ s, there exists another rational number q such that r $<$ q $<$ s

\subsubsection{Divisibility}

If n,d $\in \mathbb{Z}$, d $\neq$ 0, then n is \underline{divisible} by d if and only if there exists some k $\in \mathbb{Z}$ such that n = kd

Alternatively, we say
\begin{itemize}
\item n is a \underline{multiple} of d
\item d is a \underline{factor} of n
\item d is a \underline{divisor} of n
\item d \underline{divides} n
\end{itemize}

The notation d$|$n is used to represent the predicate "d divides n"

If n is not divisible by d, we write d$\nmid$n

\underline{Theorem:}
Every integer n$>$1 can be written as a product of primes.

\underline{Proof:}
Suppose the theorem is false. Then there exists an integer n$>$1 that is not a product of primes.

Choose the smallest number \textit{n}. Either n is prime or n is composite

\begin{itemize}
\item If n is prime; then n is trivially a product of primes
\item If n is composite: then n = rs for some positive integers r and s, where r $\neq$ 1 and s $\neq$ 1. This implies that 1 $<$ r $<$ n and 1 $<$ s $<$ n
\end{itemize}

Because we chose n to be the smallest integer greater than 1 that is not a product of primes, both r and s (which are smaller than n) \underline{must} be products of primes.

Therefore, n = rs is a product of primes also.

So, regardless of whether n is prime or composite, we find that n is a product of primes.

This contradicts our choice of n

Therefore, every integer n $>$ 1 can be written as a product of primes.

\subsubsection{Prime Factorisation}

Given any integer n $>$ 1, there exists
\begin{itemize}
\item A positive integer \textit{k}
\item Distinct primes $p_{1}$. $p_{2}$, ..., $p_{k}$
\item Distinct primes $e_{1}$. $e_{2}$, ..., $e_{k}$ such that \\
n = $(p_{1})^{(e_{1})}(p_{2})^{(e_{2})}(p_{3})^{(e_{3})} ... (p_{k})^{(e_{k})}$
\end{itemize}

and any other expression of n as a product of primes is identical to this, except perhaps for the order in which the terms are written

\subsubsection{Floor and Ceiling}

Given any x $\in \mathbb{R}$, the \underline{floor} of x, denoted $\lfloor$x$\rfloor$, is the unique integer n such that
n $\leq$ x $<$ n+1

Given any x $\in \mathbb{R}$, the \underline{ceilling} of x, denoted $\lceil$x$\rceil$, is the unique integer n such that
n - 1 $<$ x $\leq$ n+1

\subsubsection{The quotient-remainder theorem}

Given any integer n and a postiive integer d , there exists \underline{unique} integers q and r such that
\begin{itemize}
\item n = dq + r and Q $\leq$ r $<$ d
\end{itemize}

For integers \textit{a} and \textit{b}, and a positive integer \textit{d}, we say a is \underline{congruent} to b modulo d and write
\begin{itemize}
\item a $\equiv$ b (mod d)
\end{itemize}
if and only if d $|$ (a,b)

If a and b are not congruent modulo d, we write a /$\equiv$ b (mod d)
[don't know how to represent this]

\underline{Facts:}
\begin{itemize}
\item If n = dq + r, then n $\equiv$ r (mod d)
\item n $\equiv$ 0 (mod d) if and only if d$|$n
\end{itemize}

\subsection{Greatest common divisor}

For integers \textit{a,b} $\in \mathbb{Z}$, not both zero, the \underline{greatest common divisor} of a and b, denoted gcd(a,b), is the
integer \textit{d} which satisfies the following two properties:

\begin{itemize}
\item d $|$ a and d $|$ b
\item for all c $\in \mathbb{Z}$, if c $|$ a and c $|$ b then c $\leq$ d
\end{itemize}

Thus, d is the largest integer for which d $|$ a and d $|$ b

If gcd(a,b) = 1, then a and b have no common factors other than $\pm1$ and we call a and b coprime or relatively prime

What happens if a = b = 0?

For every d $\in \mathbb{Z}$, d satisfies d $|$ 0 (d $\neq$ 0) since 0 = d.0.
Thus, there is no greatest common divisor since there is no greatest integer.

However, if b $>$ 0, what is gcd(0,b)?
\begin{itemize}
\item d $|$ 0 for all d $\in \mathbb{Z}$ (d $\neq$ 0)
\item the greatest divisor of b is b
\end{itemize}

Thus, the greatest \underline{common} divisor is 1 \\

\underline{Fact:} If b $>$ 0 then gcd(0,b) = b \\

\underline{Fact:} If \textit{a} and \textit{b} are integers with b $\neq$ 0 and if \textit{q} and \textit{r} are integers such that:
\begin{align}
a = bq + r
\end{align}
then gcd(a,b) = gcd(b,r)

\section{Logical Equivalences}

Given any statement variables '\textit{p}', '\textit{q}' and '\textit{r}', a tautology '\textbf{t}' and contradiction '\textbf{c}', the following
logical equivalences hold.


\subsection{Commutative laws}
\begin{itemize} 
\item \textit{p} $\wedge$ \textit{q} $\equiv$ \textit{q} $\wedge$ \textit{p}
\item \textit{p} $\lor$ \textit{q} $\equiv$ \textit{q} $\lor$ \textit{p}
\end{itemize}

\subsection{Associative laws}
\begin{itemize}
\item (\textit{p} $\wedge$ \textit{q}) $\wedge$ \textit{r} $\equiv$ \textit{p} $\wedge$ (\textit{q} $\wedge$ \textit{r})
\item (\textit{p} $\lor$ \textit{q}) $\lor$ \textit{r} $\equiv$ \textit{p} $\lor$ (\textit{q} $\lor$ \textit{r})
\end{itemize}

\subsection{Distributive laws}
\begin{itemize}
\item \textit{p} $\wedge$ (\textit{q} $\lor$ \textit{r}) $\equiv$ (\textit{p} $\wedge$ \textit{q}) $\lor$ (\textit{p} $\wedge$ \textit{r})
\item \textit{p} $\lor$ (\textit{q} $\wedge$ \textit{r}) $\equiv$ (\textit{p} $\lor$ \textit{q}) $\wedge$ (\textit{p} $\lor$ \textit{r})
\end{itemize}

\subsection{Identity laws}
\begin{itemize}
\item \textit{p} $\wedge$ \textbf{t} $\equiv$ \textit{p}
\item \textit{p} $\lor$ \textbf{c} $\equiv$ \textit{p}
\end{itemize}

\subsection{Negation laws}
\begin{itemize}
\item \textit{p} $\lor$ $\sim$\textit{p} $\equiv$ \textbf{t}
\item \textit{p} $\wedge$ $\sim$\textit{p} $\equiv$ \textbf{c}
\end{itemize}

\subsection{Double negative laws}
\begin{itemize}
\item $\sim$($\sim$\textit{p}) $\equiv$ \textit{p}
\end{itemize}

\subsection{Idempotent laws}
\begin{itemize}
\item \textit{p} $\lor$ \textit{p} $\equiv$ \textit{p}
\item \textit{p} $\wedge$ \textit{p} $\equiv$ \textit{p}
\end{itemize}

\subsection{Universal bound laws}
\begin{itemize}
\item \textit{p} $\lor$ \textbf{t} $\equiv$ \textbf{t}
\item \textit{p} $\wedge$ \textbf{c} $\equiv$ \textbf{c}
\end{itemize}

\subsection{De Morgan's laws}
\begin{itemize}
\item $\sim$(\textit{p} $\wedge$ \textit{q}) $\equiv$ $\sim$\textit{p} $\lor$ $\sim$\textit{q}
\item $\sim$(\textit{p} $\lor$ \textit{q}) $\equiv$ $\sim$\textit{p} $\wedge$ $\sim$\textit{q}
\end{itemize}

\subsection{Absorption laws}
\begin{itemize}
\item \textit{p} $\lor$ (\textit{p} $\wedge$ \textit{q}) $\equiv$ \textit{p}
\item \textit{p} $\wedge$ (\textit{p} $\lor$ \textit{q}) $\equiv$ \textit{p}
\end{itemize}

\subsection{Negations of t and c}
\begin{itemize}
\item $\sim$\textbf{t} $\equiv$ \textbf{c}
\item $\sim$\textbf{c} $\equiv$ \textbf{t}
\end{itemize}


\end{document}