\documentclass{article}

\usepackage{amsmath}
\usepackage{amssymb}
\usepackage{graphicx}
\title{MATH1061 Course Notes}
\author{Paddy Maher}

\begin{document}
\maketitle
\newpage

\section{Logic}
\subsection{Logical Connectives}
\subsubsection{Basic logical connectives}
For a given logical statement come logical connectives. Basic logical connectives include:
\begin{itemize}
\item \textbf{not} = $\sim$
\item \textbf{and} = $\wedge$
\item \textbf{or} = $\lor$
\item \textbf{exclusive or} = $\oplus$
\end{itemize}

\subsubsection{Logical Equivalence}
Given two statement forms, you can show that they are logically equivalent by using a truth table or by using 
the laws of logical equivalence. \\
The logical equivalence between two statements is demonstrated by the symbol $\equiv$

[insert truth table]

\subsubsection{Conditional logical connectives}
Logical connectives and equivalences, for given truth statements '\textit{p}' and '\textit{q}'
\begin{itemize}
\item \textbf{if .. then} = $\rightarrow$
\item \textbf{if and only if} = $\leftrightarrow$
\item \textit{p} $\rightarrow$ \textit{q} $\equiv$ 
$\sim$ \textit{p} $\lor$ \textit{q} $\equiv$ 
$\sim$ \textit{q} $\rightarrow$ $\sim$ \textit{p}
\end{itemize}

\subsubsection{Order of Operations}
\begin{enumerate}
\item $\sim$
\item $\wedge$ and $\lor$, use parentheses to specify. If no parentheses given, work from left to right.
\item  $\rightarrow$ and $\leftrightarrow$, use parentheses to specify. If no parentheses given, work from left to right.
\end{enumerate}

\subsection{Necessary and sufficient conditions}
For given truth statements '\textit{p}' and '\textit{q}':
\begin{itemize}
\item \textit{p} is a necessary condition for \textit{q} means "if $\sim$\textit{p} then $\sim$\textit{q}" or
equivalently "if \textit{q} then \textit{p}" or "\textit{q} only if \textit{p}".
\item \textit{p} is a sufficient condition for \textit{q} means "if \textit{p} then \textit{q}" or equivalently
"\textit{q} if \textit{p}"
\end{itemize}

\subsection{Definitions}
\subsubsection{Tautology and contradictions}
\begin{itemize}
\item A \underline{tautology} is a statement form which always takes truth values "\textbf{true}" for all possible
truth values of its variables.
\item A \underline{contradiction} is a statement form which always takes truth values "\textbf{false}" for all possible
truth values of its variables.
\end{itemize}

\subsubsection{Contrapositive}
For given truth statements '\textit{p}' and '\textit{q}' \\
The \underline{contrapositive} of \textit{p} $\rightarrow$ \textit{q} is $\sim$\textit{q} $\rightarrow$ $\sim$\textit{p}
\begin{itemize}
\item These are logically equivalent: \\
\textit{p} $\rightarrow$ \textit{q} $\equiv$  $\sim$\textit{q} $\rightarrow$ $\sim$\textit{p}
\end{itemize}

\subsubsection{Biconditional}
For given truth statements '\textit{p}' and '\textit{q}' \\
The \underline{biconditional} of  \textit{p} and \textit{q}, denoted \textit{p} $\leftrightarrow$ \textit{q},
 is defined by the following truth table:

[insert truth table]

\subsection{Arguments}

\subsubsection{Premises}

Given a collection of statements '$p_1 , p_2, ... , p_n$' (called \textbf{premises}) and another statement 'q' (called the conclusion),
an '\textit{argument}' is the assertion that the conjunction of the premisees implies the conclusion.

\begin{align*}
& p_1 \\ 
& p_2 \\
& ... \\
& p_n \\
& \therefore q
\end{align*}

\subsubsection{Arguments; validity and invalidity}

\underline{Definition; valid argument}
An argument is \textbf{valid} if whenever all of the premises are true, the conclusion is also true. 

Thus, an argument is valid if ($p_1 \wedge p_2 \wedge ... \wedge p_n \rightarrow q$) is a tautology.

\underline{Definition; invalid argument}
An argument is \textbf{invalid} if it is possible to have a situation in which all of the premises are true but the conclusion is false.

We can check whether an argument is valid or invalid using a truth table.

\subsubsection{Rules of Inference}

Modus Ponens 
\begin{align*}
& p \rightarrow q \\ 
& p \\
& \therefore q
\end{align*}

Modus Tollens
\begin{align*}
& p \rightarrow q \\ 
& \sim q \\
& \therefore \sim p
\end{align*} \\

Generalisation
\begin{align*}
& p \\ 
& \therefore p \lor q \\ \\
& q \\ 
& \therefore p \lor q 
\end{align*} \\

Specialisation
\begin{align*}
& p \wedge q \\ 
& \therefore p \\ \\
& p \wedge q \\ 
& \therefore q 
\end{align*} \\

Conjunction
\begin{align*}
& p \\ 
& q \\
& \therefore p \wedge q
\end{align*} \\

Elimination
\begin{align*}
& p \lor q \\ 
& \sim q \\
& \therefore p \\ \\
& p \lor q \\ 
& \sim p \\
& \therefore q
\end{align*} \\

Transitivity
\begin{align*}
& p \rightarrow q \\ 
& q \rightarrow r \\
& \therefore p \rightarrow r
\end{align*} \\

Proof by Division into cases
\begin{align*}
& p \lor q \\ 
& p \rightarrow q \\
& q \rightarrow r \\
& \therefore r
\end{align*} \\

Contradiction Rule
\begin{align*}
& \sim p \rightarrow (contradiction) \\
& \therefore p
\end{align*} \\

\subsubsection{Alternative method for determining validity}

If an argument is \textit{'invalid'} then there is a situation where all the premises are true but the conclusion is false.

Attempt to see whether this is possible. To do this, look for truth values which make all premises true yet the conclusion is false.

If such truth values can be found, then the argument is \textit{invalid} \\

Summary of this method:
\begin{itemize}
\item Try to make all the premises true and the conclusion false
\item If this can be done, then the argument is \underline{invalid}
\item On the other hand, if this is \textbf{impossible} to do, then the argument is \underline{valid}
\end{itemize}

\underline{Checks for Validity}
\begin{itemize}
\item Use a truth table
\item Use rules of inference
\item Attempt to find turht values that make all premises true but the conclusion false.
\end{itemize}

\subsubsection{Predicates and domains}

A \underline{predicate} is a sentence that contains finitely many variables, and which becomes a statement if the variables are given
specific values. \\

The \underline{domain} of each variable in a predicate is the set of all possible values that may be assigned to it. \\

The \underline{truth set} of a predicate P(x) is the set of all values in the domain that, when assigned to x, make P(x) a true statement.

\underline{Common Domains}
\begin{itemize}
\item Integers: $\mathbb{Z}$ = [ ..., -3, -2, -1, 0, 1, 2, 3, .. ]
\item Positive integers: $\mathbb{Z}^{+}$ = [ 1, 2, 3, ... ]
\item Non-negative integers: $\mathbb{Z}^{non-neg}$ = [ 0, 1, 2, 3, ... ]
\item Natural numbers: $\mathbb{N}$ = [ 1, 2, 3, ... ]
\item Rational numbers: $\mathbb{Q}$ = [ $\frac{a}{b}$ $|$ a, b $\in \mathbb{Z} \wedge$ b $\neq$ 0 ]
\item Real numbers: $\mathbb{R}$ = entire number line
\end{itemize}

\subsection{Quantifiers}

\subsubsection{Universal and Existental quantifiers}

\underline{The Universal quantifier}

The symbol '$\forall$' denotes \textit{"for all"} and is called the \textbf{universal quantifier}

Let Q(x) be a predicate and \textbf{D} be the domain of x. The \textbf{universal statement}

$\forall$ x $\in$ \textbf{D}, Q(x)

is true if and only if Q(x) is true for every x in \textbf{D}. It is false if and only if Q(x) is false for at least one x in \textbf{D} \\

\underline{The Existental Quantifier}

The symbol '$\exists$' denotes "there exists" and is called the \textbf{existential quantifier}

Let Q(x) be a predicate and \textbf{D} be the domain of x. The \textbf{existential statement}

$\exists$ x $\in$ \textbf{D} such that Q(x)

is true if and only if Q(x) is true for at least one x in \textbf{D}. It is false if and only if Q(x) is false for every x in \textbf{D}

\subsubsection{Negation of Quantified Statements}

\underline{Universal Statement:}

$\forall$ x $\in$ \textbf{D}, Q(x)

The negation of this statement is logically equivalent to:

$\exists$ x $\in$ \textbf{D} such that $\sim$ Q(x) \\

\underline{Existential Statement:}

$\exists$ x $\in$ \textbf{D} such that Q(x) 

The negation of this statement is logically equivalent to:

$\forall$ x $\in$ \textbf{D}, $\sim$ Q(x) \\

\underline{Universal Conditional Statement}

$\forall$ x $\in$ \textbf{D} if P(x) then Q(x)

The negation of this statment is logically equivalent to:

$\exists$ x $\in$ \textbf{D} such that $\sim$ if P(x) then Q(x) 

which is;

$\exists$ x $\in$ \textbf{D} such that P(x) $\wedge \sim$Q(x)

\subsection{Multiple quantifiers}

\subsubsection{Intro to multiple quantifers}

The predicate x $\leq$ y for real numbers \textit{x} and \textit{y} involves more than one variable.

Notation such as P(x,y) is used to denote such predicates.

Such predicates often appear in statments that involve more than one quantifier \\

In order to establish the truth of a statment of the form:

$\forall$ x $\in$ \textbf{D}, $\exists$ y $\in$ \textbf{D} such that P(x,y)

One must allow another to pick whatever element x $\in$ \textbf{D} they wish, and then must preceed with finding an element
y $\in$ \textbf{E} which makes P(x,y) true. \\

In order to establish the truth of a statment of the form:

$\exists$ x $\in$ \textbf{D} such that $\forall$ y $\in$ \textbf{D}, P(x,y)

One must find one particular x $\in$ \textbf{D} which makes P(x,y) true no matter which y $\in$ \textbf{D} might be chosen
for you.

\subsubsection{Negation of statements with multiple quantifers}

The statement: 

$\forall$ x $\in$ \textbf{D}, $\exists$ y $\in$ \textbf{E} such that P(x,y)

Negates to: 

$\exists$ x $\in$ \textbf{D} such that $\sim$ ($\exists$ y $\in$ \textbf{E} such that P(x,y))

Which is:

$\exists$ x $\in$ \textbf{D} such that $\forall$ y $\in$ \textbf{E} $\sim$ P(x,y) \\


The statement: 

$\exists$ x $\in$ \textbf{D} such that $\forall$ y $\in$ \textbf{E}, P(x,y)

Negates to:

$\forall$ x $\in$ \textbf{D}, $\sim$ ($\forall$ y $\in$ \textbf{E}, P(x,y))

Which is:

$\forall$ x $\in$ \textbf{D}, $\exists$ y $\in$ \textbf{E} such that $\sim$ P(x,y))

\section{Proofs and Number Theory}

\subsection{Proofs}

\subsubsection{Even and Odd}

An integer 'n' is \underline{even} if and only if 'n' is twice some integer.
\begin{itemize}
\item That is: n is even $\leftrightarrow \exists k \in \mathbb{Z}$ such that n = 2k
\end{itemize}

An integer 'n' is \underline{odd} if and only if 'n' is twice some integer.
\begin{itemize}
\item That is: n is odd $\leftrightarrow \exists k \in \mathbb{Z}$ such that n = 2k + 1
\end{itemize}

\section{Logical Equivalences}

Given any statement variables '\textit{p}', '\textit{q}' and '\textit{r}', a tautology '\textbf{t}' and contradiction '\textbf{c}', the following
logical equivalences hold.


\subsection{Commutative laws}
\begin{itemize} 
\item \textit{p} $\wedge$ \textit{q} $\equiv$ \textit{q} $\wedge$ \textit{p}
\item \textit{p} $\lor$ \textit{q} $\equiv$ \textit{q} $\lor$ \textit{p}
\end{itemize}

\subsection{Associative laws}
\begin{itemize}
\item (\textit{p} $\wedge$ \textit{q}) $\wedge$ \textit{r} $\equiv$ \textit{p} $\wedge$ (\textit{q} $\wedge$ \textit{r})
\item (\textit{p} $\lor$ \textit{q}) $\lor$ \textit{r} $\equiv$ \textit{p} $\lor$ (\textit{q} $\lor$ \textit{r})
\end{itemize}

\subsection{Distributive laws}
\begin{itemize}
\item \textit{p} $\wedge$ (\textit{q} $\lor$ \textit{r}) $\equiv$ (\textit{p} $\wedge$ \textit{q}) $\lor$ (\textit{p} $\wedge$ \textit{r})
\item \textit{p} $\lor$ (\textit{q} $\wedge$ \textit{r}) $\equiv$ (\textit{p} $\lor$ \textit{q}) $\wedge$ (\textit{p} $\lor$ \textit{r})
\end{itemize}

\subsection{Identity laws}
\begin{itemize}
\item \textit{p} $\wedge$ \textbf{t} $\equiv$ \textit{p}
\item \textit{p} $\lor$ \textbf{c} $\equiv$ \textit{p}
\end{itemize}

\subsection{Negation laws}
\begin{itemize}
\item \textit{p} $\lor$ $\sim$\textit{p} $\equiv$ \textbf{t}
\item \textit{p} $\wedge$ $\sim$\textit{p} $\equiv$ \textbf{c}
\end{itemize}

\subsection{Double negative laws}
\begin{itemize}
\item $\sim$($\sim$\textit{p}) $\equiv$ \textit{p}
\end{itemize}

\subsection{Idempotent laws}
\begin{itemize}
\item \textit{p} $\lor$ \textit{p} $\equiv$ \textit{p}
\item \textit{p} $\wedge$ \textit{p} $\equiv$ \textit{p}
\end{itemize}

\subsection{Universal bound laws}
\begin{itemize}
\item \textit{p} $\lor$ \textbf{t} $\equiv$ \textbf{t}
\item \textit{p} $\wedge$ \textbf{c} $\equiv$ \textbf{c}
\end{itemize}

\subsection{De Morgan's laws}
\begin{itemize}
\item $\sim$(\textit{p} $\wedge$ \textit{q}) $\equiv$ $\sim$\textit{p} $\lor$ $\sim$\textit{q}
\item $\sim$(\textit{p} $\lor$ \textit{q}) $\equiv$ $\sim$\textit{p} $\wedge$ $\sim$\textit{q}
\end{itemize}

\subsection{Absorption laws}
\begin{itemize}
\item \textit{p} $\lor$ (\textit{p} $\wedge$ \textit{q}) $\equiv$ \textit{p}
\item \textit{p} $\wedge$ (\textit{p} $\lor$ \textit{q}) $\equiv$ \textit{p}
\end{itemize}

\subsection{Negations of t and c}
\begin{itemize}
\item $\sim$\textbf{t} $\equiv$ \textbf{c}
\item $\sim$\textbf{c} $\equiv$ \textbf{t}
\end{itemize}


\end{document}